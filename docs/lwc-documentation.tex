\environment lwc-documentation
% lua-widow-control
% https://github.com/gucci-on-fleek/lua-widow-control
% SPDX-License-Identifier: MPL-2.0+ OR CC-BY-SA-4.0+
% SPDX-FileCopyrightText: 2021 gucci-on-fleek

\useURL[projecturl][https://github.com/gucci-on-fleek/lua-widow-control]

\def\lwc/{\sans{lua\allowbreak-widow\allowbreak-control}}
\def\Lwc/{\sans{Lua\allowbreak-widow\allowbreak-control}}
\def\lwcenable/{\tex{Lua\allowbreak Widow\allowbreak Control\allowbreak Enable}}
\def\lwcdisable/{\tex{Lua\allowbreak Widow\allowbreak Control\allowbreak Disable}}

\usemodule[vim]

\definevimtyping[TEX][syntax=tex, lines=split, escape=comment]
\definevimtyping[LUA][syntax=LUA, lines=split, escape=comment]

\startdocument[
    title=lua-widow-control,
    author=Max Chernoff,
    version=0.0.0,
]

\Lwc/ is a Plain~\TeX/\LaTeX/\ConTeXt\ package that removes most widows and orphans from a document, \emph{without} stretching any glue or shortening any pages. It does so by automatically lengthening a paragraph on a page where a widow or orphan would otherwise occur. 

\warning{This is a \bold{preliminary} release of \lwc/. So far, it has only been tested with simple text---it has \emph{not} been tested with tables, floats, footnotes, or graphics. In testing, \lwc/ has sometimes randomly discarded paragraphs without any warning or indication. This should not happen anymore (and please report it if it does), but I would still recommend against using \lwc/ on exams, contracts, or other important documents.}

\section{Installation}

Eventually, \lwc/ should be on \acronym{CTAN} and \TeX{}Live; until then, you will need to manually install the package. The procedure should be fairly similar regardless of your \acronym{OS} or \TeX{}  distribution, or format.

\subsection{Steps}
\startitemize[N]
    \item \goto{Download}[url(projecturl)] \lwc/.
    \item Copy \type{lua-widow-control.lua} to \type{TEXMFLOCAL/tex/generic/lua-widow-control/}
    \item Copy \type{lua-widow-control.sty} to \type{TEXMFLOCAL/tex/latex/lua-widow-control/}
    \item Copy \type{t-lua-widow-control.mkxl} to \type{TEXMFLOCAL/tex/context/third}
    \item Refresh the filename database with \type{mtxrun --generate} (\ConTeXt), \type{mktexlsr} (\TeX{}Live), or \type{initexmf --update-fndb} (Mik\TeX).
\stopitemize

\section{Loading the Package}

\subsection{Plain \TeX}

\inlineTEX{\input lua-widow-control.sty}

\subsection{\LaTeX}

\inlineTEX{\usepackage{lua-widow-control}}

\subsection{\ConTeXt}

\inlineTEX{\usemodule[lua-widow-control]}

\section{Usage}

\Lwc/ is enabled as soon as you load it. If you wish, you can disable it with \lwcdisable/. Later, you can reenable it with \lwcenable/

\warning{If \lwc/ is already disabled, running \lwcdisable/ will throw a fatal error. Likewise, running \lwcenable/ while \lwc/ is already enabled may cause unpredictable behavior.

This will be fixed in a future update}

\section{Configuration}

There aren't very many options available yet for \lwc/. Right now, the only configurable option is the \tex{emergencystretch} used when stretching a paragraph. The default value is 3\,em.

There isn't a proper user interface quite yet, but you can adjust this by placing \inlineTEX{\directlua{lwc.emergency_stretch = tex.sp("99pt")}} after you load the package.

Admittedly, this is a pretty hostile user interface. This will be fixed in a future update.

\section{License}

\Lwc/ is licensed under the \goto{\emph{Mozilla Public License}, version 2.0}[url(https://www.mozilla.org/en-US/MPL/2.0/)] or greater. The documentation is additionally licensed under \goto{\acronym{CC-BY-SA}, version 4.0}[url(https://creativecommons.org/licenses/by-sa/4.0/legalcode)] or greater.

\page
\setuplayout[
    width=middle,
    backspace=1in,
    height=9.25in,
]
\section{Implementation}

\setupbodyfont[10pt]

\filename{lua-widow-control.lua}

\typeLUAfile{../source/lua-widow-control.lua}

\filename{lua-widow-control.sty}

\typeTEXfile{../source/lua-widow-control.sty}

\filename{t-lua-widow-control.mkxl}

\typeTEXfile{../source/t-lua-widow-control.mkxl}

\stopdocument
