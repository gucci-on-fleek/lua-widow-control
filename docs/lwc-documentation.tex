%!TeX program = context
\environment lwc-documentation
% lua-widow-control
% https://github.com/gucci-on-fleek/lua-widow-control
% SPDX-License-Identifier: MPL-2.0+ OR CC-BY-SA-4.0+
% SPDX-FileCopyrightText: 2021 gucci-on-fleek

\usemodule[lua-widow-control]
\usebtxdataset[lwc-documentation.bib]

\useURL[projecturl][https://github.com/gucci-on-fleek/lua-widow-control]
\useURL[download][https://github.com/gucci-on-fleek/lua-widow-control/releases/latest][][latest release]

\def\lwc/{\sans{lua-\allowbreak widow-\allowbreak control}}
\def\Lwc/{\sans{Lua-\allowbreak widow-\allowbreak control}}
\def\lwcenable/{\tex{Lua\allowbreak Widow\allowbreak Control\allowbreak Enable}}
\def\lwcdisable/{\tex{Lua\allowbreak Widow\allowbreak Control\allowbreak Disable}}
\def\estretch/{\tex{emergency\allowbreak stretch}}
\def\openalty/{\tex{output\allowbreak penalty}}

\def\titlecite[#1]{\cite[title][#1]\cite[#1]}

\usemodule[vim]
\definevimtyping[TEX][syntax=tex, lines=split, escape=comment]
\definevimtyping[LUA][syntax=LUA, lines=split, escape=comment]

\input lwc-sample

\startdocument[
    title=lua-widow-control,
    author=gucci-on-fleek,
    version=0.0.0,
]

\Lwc/ is a Plain~\TeX/\LaTeX/\ConTeXt\ package that removes most widows and orphans from a document, \emph{without} stretching any glue or shortening any pages. It does so by automatically lengthening a paragraph on a page where a widow or orphan would otherwise occur. 

\warning{\Lwc/ uses Lua callbacks to modify the internal structure of each page. This typically causes no issues; however, in rare scenarios, it may \bold{delete or duplicate} random paragraphs or lines. All known bugs that cause this issue have been corrected; nevertheless, I recommend against using this package for exams, contracts, or any other document where textual changes would be disastrous. This package should be sufficiently stable for use in typical documents.}

\section{Quick Start}
Install \lwc/ by unzipping the \from[download] in your \type{TEXMFLOCAL/} directory. Rebuild the filename database. Place \inlineTEX{\usepackage{lua-widow-control}} in the preamble of your document.

For a more detailed explanation of the steps involved and a description of some of the finer points involved with using \lwc/, keep reading.

\subject{Contents}
\placecontent[criterium=all]

\section{Widows and Orphans}
\subsection{Widows}
Widows occur when when the last line of a paragraph is on the page after the rest of the paragraph. Widows are quite undesirable since it is usually quite distracting for the reader to find a stray line on the top of a page. Wherever possible, this should be avoided.

\subsection{Orphans}
Orphans are when the first line of a paragraph occurs on the page before the remainder of they paragraph. They are not nearly as distracting for the reader, but they are still far from ideal.

\subsection{Visual}
For a visual comparison of widows and orphans, please see \in{Figure}[tab:widow].

\startplacefigure[location={here, top, bottom}, title={A visual comparison of widows and orphans.}, reference=tab:widow]
    \getbuffer[widow-orphan]
\stopplacefigure

\section{\TeX's Default Behavior}

It is tricky to understand how \lwc/ works if you aren't familiar with how \TeX{} breaks pages. The \titlecite[texbook] is the best reference for this; however, it goes into much more detail than is required to understand \lwc/. Therefore, a \emph{very} simplified (and likely incorrect) explanation follows. 

\TeX{} fills the page with lines and other objects until the next object will no longer fit. Once no more objects will fit, \TeX{}, will align the bottom of the last line with the bottom of the page by stretching any vertical spaces.

However, some objects have penalties attached. These penalties make \TeX{} treat the object as if it is longer or shorter for the sake of page breaking. By default, \TeX{} assigns a penalties to the first and last lines of a paragraph (orphans and widows). This makes \TeX{} treat them as if they are larger or smaller than their actual size such that \TeX{} tends not to break them up.

One important note: once \TeX{} begins breaking a page, it only breaking a page. That is, it will never go back and modify any content that has previously been added to its \emph{main vertical list}. That is, page breaking is a localized algorithm, without any backtracking.

\section{Previous Work}
There have been a few previous attempts to improve upon \TeX's builtin widow and orphan-handling abilities; however, none of these have been able to automatically remove widows and orphans without stretching any glue or shortening any pages.

\titlecite[tugboat-strategies] and \titlecite[tugboat-widows] both begin with comprehensive discussions of the methods of preventing widows and orphans. They both agree that widows and orphans are bad and ought to be avoided; however, they each differ in solutions. \italic{Strategies}\cite[tugboat-strategies] proposes an output routine that reduces the length of facing pages by 1 line if necessary to remove widows and orphans while \italic{Managing}\cite[tugboat-widows] proposes that the author manually rewrites or adjusts the \tex{looseness} when needed.

\titlecite[widow-assist] contains a file \type{widow-assist.lua} that automatically detects which paragraphs can be safely shortened or lengthened by 1 line. \titlecite[widows-and-orphans] alerts the author to the pages that contain widows or orphans. Combined, these packages make it very simple for the author to quickly remove widows and orphans by adjusting the \tex{looseness} values; however, it still requires the author to make manual source changes after each revision. 

\Lwc/ is essentially just a combination of \type{widow-assist.lua}\cite[widow-assist] and \sans{widows-and-orphans}:\cite[widows-and-orphans] when the \openalty/ shows that a widow or orphan occurred, Lua is used to find a stretchable paragraph. What \lwc/ adds on top of these packages is automation: \lwc/ eliminates the requirement for any manual adjustments.

\section{Demonstration}

Although \TeX{}'s page breaking algorithm is quite simple, it can lead to some fairly complex behaviors in regards to orphans and widows. The usual choices are to either ignore them, stretch some glue, or shorten the page. \in{Table}[tab:demo] on the next page shows a visual demonstration of some of these behaviors and how \lwc/ differs from the defaults.

\startTEXpage[
    align=normal,
    width=100cm,
    autowidth=force,
    offset=5pt,
    pagestate=start
]
    \veryraggedcenter
    \setupTABLE[row][first][style=\ssbf, align=middle]
    \setupTABLE[frame=off, offset=5pt]
    \startTABLE
        \NC Ignore
        \NC Shorten
        \NC Stretch
        \NC \Lwc/ \NC\NR

        \NC \typesetbuffer[ignore][frame=on,page=1]
        \NC \typesetbuffer[shorten][frame=on,page=1]
        \NC \typesetbuffer[stretch][frame=on,page=1]
        \NC \typesetbuffer[lwc][frame=on,page=1]
        \NC\NR 

        \NC \typesetbuffer[ignore][frame=on,page=2]
        \NC \typesetbuffer[shorten][frame=on,page=2]
        \NC \typesetbuffer[stretch][frame=on,page=2]
        \NC \typesetbuffer[lwc][frame=on,page=2]
        \NC\NR

        \NC \processTEXbuffer[ignore-code]
        \NC \processTEXbuffer[shorten-code]
        \NC \processTEXbuffer[stretch-code]
        \NC \processTEXbuffer[lwc-code]
        \NC\NR
    \stopTABLE

    \placefloatcaption[table][reference=tab:demo, title={A visual comparison of various automated widow handling techniques.}]
\stopTEXpage

\subsection{Ignore}
As you can see, the line of this last paragraph is on a separate page from the rest of the paragraph, creating a widow. This is usually pretty distracting for the reader, so it is best avoided.

\subsection{Shorten}
This page did not leave any widows, but it did shorten the previous page by 1~line. Sometimes this is alright, but usually it looks bad because each page will have different text\-block heights. This can make the pages look quite uneven, especially when typesetting with columns.

\subsection{Stretch}
This page also has no widows and it has a flushed bottom margin. However, the space between each paragraph had to be stretched. 

If this page had many equations, headings, and other elements with natural space between them, the stretched out space would be much less noticeable. \TeX{} was designed for mathematical typesetting, so it makes sense that this is its default behavior. However, in a page with mostly text, these paragraph gaps can look unsightly. 

In addition, this method is incompatible with typesetting on a grid, since there is no flexible glue on the page.

\subsection{\lwc/}
\Lwc/ has none of these issues: it eliminates the widows in a document while keeping a flushed bottom margin and constant paragraph spacing. 

To do so, \lwc/ lengthened the second paragraph by one line. If you look closely, you can see that this stretched the interword spaces. This stretching is noticeable when typesetting in a narrow text block, but it becomes nearly imperceptible with larger widths.

\Lwc/ automatically finds the \quotation{best} paragraph to stretch, so the increase in interword spaces should almost always be minimal.

\section{Installation}

Eventually, \lwc/ should be on \acronym{CTAN} and \TeX{}Live; until then, you will need to manually install the package. The procedure should be fairly similar regardless of your \acronym{OS}, \TeX{}  distribution, or format.

\subsection{Steps}
\startitemize[N, packed]
    \item Download the \from[download] of \lwc/.
    \item Unzip or un\type{tar} the release into your \type{TEXMFLOCAL/} directory. (You can find its location by running \type{kpsewhich --var-value TEXMFHOME} in a terminal)
    \item Refresh the filename database: \startitemize[1, packed]
        \item \ConTeXt: \type{mtxrun --generate}
        \item \TeX{}Live: \type{mktexlsr}
        \item Mik\TeX: \type{initexmf --update-fndb}
    \stopitemize
\stopitemize

\section{Loading the Package}

\subsection{Plain \TeX}

\inlineTEX{\input lua-widow-control}

\subsection{\LaTeX}

\inlineTEX{\usepackage{lua-widow-control}}

\subsection{\ConTeXt}

\inlineTEX{\usemodule[lua-widow-control]}

\section{Usage}

\Lwc/ is enabled as soon as you load it. If you wish, you can disable it with \lwcdisable/. Later, you can reenable it with \lwcenable/

\warning{If \lwc/ is already disabled, running \lwcdisable/ will throw a fatal error. Likewise, running \lwcenable/ while \lwc/ is already enabled may cause unpredictable behavior.

This will be fixed in a future update}

\section{Configuration}

There aren't very many options available yet for \lwc/. Right now, there is only a single configurable option.

\subsection{\estretch/}

You can configure the \estretch/ used when stretching a paragraph. The default value is 3\,em. \Lwc/ will only use use the \estretch/ when it cannot lengthen a paragraph in any other way, so it is fairly safe to set this to a large value. \TeX{} still accumulates badness when \estretch/ is used, so its pretty rare that a paragraph that requires any \estretch/ will actually be used on the page.

You can adjust the \estretch/ value used by \lwc/ by placing \inlineTEX{\directlua{lwc.emergency_stretch = tex.sp("999pt")}} after you load the package. In a future update, I'll add a proper user interface. 

% \section{Known Issues}
% \startitemize
%     \item
% \stopitemize

\subsection{The Algorithm}
\Lwc/ uses a fairly simple algorithm to eliminate widows and orphans. It is pretty basic, but there are a few subtleties. Please see \about[sec:implementation] for a full listing of the source code.

\subsection{Paragraph Breaking}
First, \lwc/ hooks into the paragraph breaking process.

Before a paragraph is broken by \TeX{}, \lwc/ grabs the unbroken paragraph. \Lwc/ then breaks the paragraph 1~line longer than its natural length and stores it for later. 

\Lwc/ does not interfere with how \TeX{} paragraphs into their natural length.

After \TeX{} has broken its paragraph into its natural length, \lwc/ appears once again. Before the broken paragraph is added to the main vertical list, \lwc/ tags the first and last nodes of the paragraph. These tags create a relationship between the previously-saved lengthened paragraph and the start/end of the naturally-typeset paragraph on the page.

\subsection{Page Breaking}
\Lwc/ intercepts \tex{box255} after the output routine and before it is shipped out. 

First, \lwc/ analyzes the \openalty/ of the page. If the page was broken at a widow or an orphan, the \openalty/ will have been set to the respective values of \tex{widow\allowbreak penalty} and \tex{orphan\allowbreak penalty}. If the \openalty/ is not indicative of a widow or an orphan, \lwc/ will abort and return \tex{box255} unmodified.

At this point, we know that we have a widow or an orphan on the page. Now, we iterate through the list of saved paragraphs to find the lengthened paragraph with the least demerits. Now, we traverse through the page to find the start and the end of the original paragraph so that we can splice in the lengthened paragraph. 

Finally, we pull the last line off of the page and then we place it onto the top of the \emph{recent contributions} list. Now, we can ship out the new, widow-free page.

\section{License}

\Lwc/ is licensed under the \goto{\emph{Mozilla Public License}, version 2.0}[url(https://www.mozilla.org/en-US/MPL/2.0/)] or greater. The documentation is dual-licensed under \goto{\acronym{CC-BY-SA}, version 4.0}[url(https://creativecommons.org/licenses/by-sa/4.0/legalcode)] or greater and the \acronym{MPL}.

\section{References}

\placelistofpublications

\page
\setuplayout[
    width=middle,
    backspace=1in,
    height=9.25in,
]
\section[sec:implementation]{Implementation}

\setupbodyfont[10pt]

\filename{lua-widow-control.lua}

\typeLUAfile{../source/lua-widow-control.lua}

\filename{lua-widow-control.tex}

\typeTEXfile{../source/lua-widow-control.tex}

\filename{lua-widow-control.sty}

\typeTEXfile{../source/lua-widow-control.sty}

\def\module{\tex{module}}
\filename{t-lua-widow-control.mkxl}

\typeTEXfile{../source/t-lua-widow-control.mkxl}

\stopdocument
